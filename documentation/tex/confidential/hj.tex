\section{劫持系统例程}
\subsection{系统调用}
x86\_64 平台上,Linux~支持系统调用的两种方式,通过 int 0x80 或 syscall 指
令。两者都是通过往一个 CPU 事先约定好的存储区域注册一个地址,当需要的时候
CPU 会从相应地址处开始执行。前者称为 IDT (\textit{Interrupt Descriptors Table}),
后者则存储与 MSR (Model Specific Registers) 的 LSTAR。我们目前只关心后者。

通过 MSR 注册系统调用的路径大致如下:

\begin{verbatim}
start_kernel() -> trap_init() -> cpu_init() -> syscall_init()
\end{verbatim}

最终调用:\verb=wrmsrl(MSR_LSTAR, system_call);= 将符号 \verb=system_call= 的地址注
册。符号 \verb=system_call= 在 \verb=arch/x86_64/kernel/enry.S= 中定义。
用户空间的库函数  (e.g, glibc),实际上就是设置好参数,譬如把系统调用号存
入 \verb=rax= 中,然后发出一条 syscall 指令,之后控制进入 
\verb=system_call=   函数中,该函数一开始做一些参数入栈、测试参数合法性等
辅助工作,如果一切顺利,最终将走到:\verb=call *sys_call_table(,%rax,8)= 
这条语句。显而易见,这里 \verb=sys_call_table= 是一个大数组,
里面存放所有系统调用对应的例程,例如该数组的第  \verb=__NR_clone= 项就存
放着 \verb=clone= 系统调用对应的例程。




